%\usepackage[nomarkers,notablist,nofiglist]{endfloat}


\documentclass[11pt]{article}
%%%%%%%%%%%%%%%%%%%%%%%%%%%%%%%%%%%%%%%%%%%%%%%%%%%%%%%%%%%%%%%%%%%%%%%%%%%%%%%%%%%%%%%%%%%%%%%%%%%%%%%%%%%%%%%%%%%%%%%%%%%%%%%%%%%%%%%%%%%%%%%%%%%%%%%%%%%%%%%%%%%%%%%%%%%%%%%%%%%%%%%%%%%%%%%%%%%%%%%%%%%%%%%%%%%%%%%%%%%%%%%%%%%%%%%%%%%%%%%%%%%%%%%%%%%%
\usepackage{eurosym}
\usepackage{amsmath}
\usepackage{amsfonts}
\usepackage{amssymb}
\usepackage{bbm}
\usepackage{color}
\usepackage{url}
\usepackage{hyperref}
\usepackage{kbordermatrix}
\usepackage{tikz}
\usepackage{multirow, array}
\usepackage{graphicx}
\usepackage{float}
\usepackage{subcaption}
\usepackage{setspace}
\usepackage[margin=1in]{geometry}
\usepackage[title]{appendix}
\usepackage[ruled,lined,algonl]{algorithm2e}

\setcounter{MaxMatrixCols}{10}
%TCIDATA{OutputFilter=LATEX.DLL}
%TCIDATA{Version=5.50.0.2960}
%TCIDATA{<META NAME="SaveForMode" CONTENT="1">}
%TCIDATA{BibliographyScheme=Manual}
%TCIDATA{LastRevised=Sunday, November 27, 2022 21:35:24}
%TCIDATA{<META NAME="GraphicsSave" CONTENT="32">}
%TCIDATA{<META NAME="DocumentShell" CONTENT="Standard LaTeX\Blank - Standard LaTeX Article">}
%TCIDATA{Language=American English}
%TCIDATA{CSTFile=40 LaTeX article.cst}
%TCIDATA{ComputeDefs=
%$V=\left( \left( 1\right) \mathbf{I}-\gamma \mathbf{C-\hat{C}}\right) ^{-1}%
%\left[ \left( \mathbf{v}-\mathbf{Cv}-\mathbf{b}\right) \circ \mathbf{DG}+%
%\mathbf{X}\right] .$
%}


\doublespacing
\usetikzlibrary{shapes}
\usetikzlibrary{arrows.meta}
\tikzset{
diagonal fill/.style 2 args={fill={rgb,255:red,118; green,219; blue,128}, path picture={
\fill[#1, sharp corners] (path picture bounding box.south east) -|
(path picture bounding box.center) -- cycle;}},
reversed diagonal fill/.style 2 args={fill=yellow, path picture={
\fill[#1, sharp corners] (path picture bounding box.center) |-
(path picture bounding box.south east) -- cycle;}}
}
\newtheorem{theorem}{Theorem}[section]
\newtheorem{acknowledgement}{Acknowledgement}[section]
\newtheorem{axiom}{Axiom}[section]
\newtheorem{case}{Case}[section]
\newtheorem{claim}{Claim}[section]
\newtheorem{conclusion}{Conclusion}[section]
\newtheorem{condition}{Condition}[section]
\newtheorem{conjecture}{Conjecture}[section]
\newtheorem{corollary}{Corollary}[section]
\newtheorem{criterion}{Criterion}[section]
\newtheorem{definition}{Definition}[section]
\newtheorem{example}{Example}[section]
\newtheorem{exercise}{Exercise}[section]
\newtheorem{lemma}{Lemma}[section]
\newtheorem{assumption}{Assumption}[section]
\newtheorem{notation}{Notation}[section]
\newtheorem{problem}{Problem}[section]
\newtheorem{proposition}{Proposition}[section]
\newtheorem{remark}{Remark}[section]
\newtheorem{solution}{Solution}[section]
\newtheorem{summary}{Summary}[section]
\SetKwData{Left}{left}
\SetKwData{This}{this}
\SetKwData{Up}{up}
\SetKwFunction{Union}{Union}
\SetKwFunction{FindCompress}{FindCompress}
\SetKwInOut{Input}{Input}
\SetKwInOut{Output}{Output}
\allowdisplaybreaks
\newenvironment{proof}[1][Proof]{\noindent\textbf{#1.} }{\ \rule{0.5em}{0.5em}}
\numberwithin{equation}{section}
\graphicspath{ {graphs/} }
\newcommand\mycaption[2]{\caption{#1\newline\small#2}}
\input{tcilatex}
\begin{document}


\section{Continuous-time Markov chain simulation}

Simulation of a continuous-time Markov chain is a common technique, which
can be found in numerous literature. We follow the standard stochastic
simulation algorithm (SSA), also known as the Gillespie algorithm, which can
be found in Gillespie (1976).

The inputs for this algorithm are the current time $t$, the maturity time $T$%
, current state of the Markov chain $\mathbf{s}_t$ and the transition rate
matrix $A^Q$.

The output for this algorithm are two sequences $\Delta_{time} =\left\lbrace
t_0,\ldots t_n \right\rbrace $ and $\Delta_{state}\left\lbrace \mathbf{s}%
_0,\ldots \mathbf{s}_n\right\rbrace $, $n\in \mathbb{N}_0$, representing the
transition time points and the states of the Markov chain, respectively. For
the time sequence $\left\lbrace t_0,\ldots t_n \right\rbrace $, we have $%
t_0=t$ denoting the current time and $t_1,\ldots, t_n<T$ denoting the time
points of state transitions. Note that if $n=0$, there is no state
transition during $\left[t,T \right]$. For the state sequence $\left\lbrace 
\mathbf{s}_0,\ldots \mathbf{s}_n\right\rbrace$, we have $\mathbf{s}_{i}$
denoting the state in the period $\left[ t_{i}, t_{i+1} \right] $ for $i\leq
n-1$ and $\mathbf{s}_n$ denoting the state in the period $\left[ t_{n}, T %
\right] $. Fig \ref{Fig:ctmc} shows a more intuitive the relationship
between the two sequences.

\begin{figure}[h]
\begin{tikzpicture}
		\draw (0,0) -- (15,0);
		\draw(0,0) node[anchor=north] {$t_0=t$} -- (0,0.2);
		\draw(0.75,1) node[anchor=south] {$\mathbf{s}_0=\mathbf{s}_t$} ;
		\draw(3,1) node[anchor=south] {$\mathbf{s}_1$} ;
		\draw(10.5,1) node[anchor=south] {$\mathbf{s}_{n-1}$} ;
		\draw(13,1) node[anchor=south] {$\mathbf{s}_{n}$} ;
		\draw(1.5,0) node[anchor=north] {$t_1$} -- (1.5,0.2);
		\draw(5,0) node[anchor=north] {$t_2$} -- (5,0.2);
		\draw(7.5,0) node[anchor=north] {... ...} ;
		\draw(10,0) node[anchor=north] {$t_{n-1}$} -- (10,0.2) ;
		\draw(11,0) node[anchor=north] {$t_{n}$} -- (11,0.2) ;
		\draw(15,0) node[anchor=north] {$T$} -- (15,0.2) ;
		%	\draw (2.5,-0.5) node[anchor=north] {$Price_{1}(t)$};
		%	\draw[red,thick,->] (10,0)--(10,2);
		%	\draw (10,2)node[anchor=south] {$K(1+c(T-t))1_{\{\tau>T\}}$};
	\end{tikzpicture}
\caption{An example of a sample path of the continuous time Markov chain}
\label{Fig:ctmc}
\end{figure}

\section{The algorithm}

Following Duffie et al. (2000), we define the following functions:%
\begin{eqnarray*}
B\left( \cdot ,\cdot ,\cdot ,\cdot ;\mathbf{s}\right) &:&\mathbb{R}%
_{+}\times \mathbb{R}_{+}\times \mathbb{R}^{3}\times \mathbb{R}%
^{3}\rightarrow \mathbb{R}^{3}, \\
C\left( \cdot ,\cdot ,\cdot ,\cdot ;\mathbf{s}\right) &:&\mathbb{R}%
_{+}\times \mathbb{R}_{+}\times \mathbb{R}^{3}\times \mathbb{R}%
^{3}\rightarrow \mathbb{R}, \\
\mathcal{B}\left( \cdot ,\cdot ,\cdot ;\mathbf{s}\right) &:&\mathbb{R}%
_{+}\times \mathbb{R}_{+}\times \mathbb{R}^{3}\rightarrow \mathbb{R}^{3}, \\
\mathcal{C}\left( \cdot ,\cdot ,\cdot ,\cdot ,\cdot ;\mathbf{s}\right) &:&%
\mathbb{R}_{+}\times \mathbb{R}_{+}\times \mathbb{R}^{3}\times \mathbb{R}%
^{3}\times \mathbb{R}^{3}\rightarrow \mathbb{R},
\end{eqnarray*}
whose closed-form expressions can be readily obtained from their ODEs and
boundary conditions/are provided in appendix. In this section, we are going
to gave recursive algorithms for computing the four parts of the bond price
conditional on the initial values of three risk factors $\mathbf{Y}_t=%
\mathbf{y}$ and a sample path of the Markov chain $\Delta_{time}$ and $%
\Delta_{state}$.

\subsection{The final redemption if no event occurs before time $T$}

Given the values of three risk factors at time $t$ and a sample path of the
Markov chain, the price of this payment at time $t$ is%
\begin{equation}
KE^{Q}\left[ \left. \exp \left\{ -\int_{t}^{T}\left( r_{u}+\kappa \lambda
_{u}\right) du\right\} \right\vert \mathbf{Y}_{t}=\mathbf{y},\Delta
_{time},\Delta _{state}\right] .  \label{p1}
\end{equation}%
\begin{algorithm}
	
	\caption{Algorithm for evaluating equation \ref{p1}}
	\Input{$K$, $\mathbf{y}$, $\Delta
		_{time}$ and $\Delta _{state}$ } 
	\Output{value of equation \ref{p1}}
	\BlankLine
	\emph{Initialization: $n \leftarrow \left| \Delta
		_{time} \right|-1 $, $\mathbf{x}\leftarrow \left(0,0,0 \right)^\intercal $, $\boldsymbol{\phi}\leftarrow \left(\kappa,1,0 \right) $ }
	\BlankLine
	\eIf{n=0}{$\mathbf{B}_0 \leftarrow B\left(t_0,T,\boldsymbol{\phi}, \mathbf{x};\mathbf{s}_0 \right) $\\$C_0\leftarrow C\left(t_0,T,\boldsymbol{\phi}, \mathbf{x};\mathbf{s}_0 \right) $ \\ \Return{$K \exp \left\lbrace \mathbf{B}_0 \cdot \mathbf{y}+ C_0 \right\rbrace$} }
	(\#\textit{in the case $n \geq 1$}){$\mathbf{B}_{n} \leftarrow H_{\mathbf{s}_{n-1},\mathbf{s}_{n}}B\left(t_n,T,\boldsymbol{\phi}, \mathbf{x};\mathbf{s}_n \right) $\\$C_n\leftarrow C\left(t_n,T,\boldsymbol{\phi}, \mathbf{x};\mathbf{s}_n \right) $\\$\mathbf{x}\leftarrow \mathbf{B}_n$\\ \For(\# \textit{if $n=1$ then this for loop will be automatically skipped}){$i\leftarrow $$n-1$ \KwTo $1$}{$\mathbf{B}_i \leftarrow H_{\mathbf{s}_{i-1},\mathbf{s}_{i}}   B\left(t_i,t_{i+1},\boldsymbol{\phi}, \mathbf{x};\mathbf{s}_i \right) $\\$C_i\leftarrow C\left(t_i,t_{i+1},\boldsymbol{\phi}, \mathbf{x};\mathbf{s}_i \right) $\\ $\mathbf{x}\leftarrow \mathbf{B}_i$}$\mathbf{B}_0 \leftarrow B\left(t_0,t_1,\boldsymbol{\phi}, \mathbf{x};\mathbf{s}_0 \right) $\\$C_0\leftarrow C\left(t_0,t_1,\boldsymbol{\phi}, \mathbf{x};\mathbf{s}_0 \right) $ \\\Return{$K\exp\left\lbrace \mathbf{B}_0\cdot \mathbf{y}+\sum_{i=0}^{n} C_i \right\rbrace $}}
	
	
	
	
	\label{algo:p1}
\end{algorithm}

\subsection{The residual face value received if an event occurs before time $%
T$}

Given the values of three risk factors at time $t$ and a sample path of the
Markov chain, the price of this payment at time $t$ is%
\begin{equation}
K\int_{t}^{T} \psi _{\mathbf{S}_{v}}E^{Q}\left[ \left. \kappa\lambda
_{v}\exp \left\{ -\int_{t}^{v}\left( r_{u}+\kappa \lambda _{u}\right)
du\right\} \right\vert \mathbf{Y}_{t}=\mathbf{y},\Delta _{time},\Delta
_{state}\right] dv.  \label{p2}
\end{equation}

\begin{algorithm}
	\caption{Algorithm for evaluating equation \ref{p2}}
\Input{$K$, $\mathbf{y}$, $\Delta
	_{time}$ and $\Delta _{state}$ } 
\Output{value of equation \ref{p2}}
\BlankLine
%\emph{Initialization: $n \leftarrow \left| \Delta
%	_{time} \right|-1 $, $\mathbf{x}\leftarrow \left(0,0,0 \right)^\intercal $, $\boldsymbol{\phi}\leftarrow \left(\kappa,1,0 \right) $ }
\BlankLine
$result\leftarrow 0$\\
\For{$j\leftarrow 1$ \KwTo $100,000$}{
\textbf{Step} 1: Draw $t^{*}\sim \text{Unif}\left(t,T \right) $
\\ \textbf{Step} 2: Truncate $\Delta_{time} $ by discarding elements that is greater than $t^*$ and denote the truncated sequence by $\Delta^*_{time} $
\\ \textbf{Step} 3: Truncate $\Delta_{state} $ to keep the first $\left| \Delta^*_{time} \right| $ elements and denote the truncated sequence by $\Delta^*_{state} $
\\ \textbf{Step} 4: Set $n\leftarrow \left|\Delta^*_{time} \right|-1 $, $\mathbf{x}_1\leftarrow \left(0,0,0 \right)^\intercal $,$\mathbf{x}_2\leftarrow \left(\kappa,0,0 \right)^\intercal $ and $\boldsymbol{\phi}\leftarrow \left(\kappa,1,0 \right) $
\\ \textbf{Step} 5: \\
\eIf{$n=0$}{
	$\mathbf{B}_0\leftarrow B\left(t_0,t^*, \boldsymbol{\phi},\mathbf{x}_1;\mathbf{s}_0\right) $\\
$C_0\leftarrow C\left(t_0,t^*, \boldsymbol{\phi},\mathbf{x}_1;\mathbf{s}_0\right) $\\
$\boldsymbol{\mathcal{B}}_0\leftarrow \mathcal{B}\left(t_0,t^*, \mathbf{x}_2;\mathbf{s}_0\right) $\\
$\mathcal{C}_0\leftarrow \mathcal{C}\left(t_0,t^*, \boldsymbol{\phi},\mathbf{x}_1,\mathbf{x}_2;\mathbf{s}_0\right) $\\
$result\leftarrow result+ \psi_{\mathbf{s}_0}\left( \mathcal{C}_0+\boldsymbol{\mathcal{B}_0 \cdot \mathbf{y}}\right) \exp\{\mathbf{B}_0\cdot \mathbf{y}+ C_0 \}$}
{
$\mathbf{B}_{n} \leftarrow H_{\mathbf{s}_{n-1},\mathbf{s}_{n}}B\left(t_n,t^*,\boldsymbol{\phi}, \mathbf{x}_1;\mathbf{s}_n \right) $\\
$C_n\leftarrow C\left(t_n,t^*,\boldsymbol{\phi}, \mathbf{x}_1;\mathbf{s}_n \right) $\\
$\boldsymbol{\mathcal{B}}_{n} \leftarrow H_{\mathbf{s}_{n-1},\mathbf{s}_{n}}\mathcal{B}\left(t_n,t^*, \mathbf{x}_2;\mathbf{s}_n \right) $\\
$\mathcal{C}_n\leftarrow \mathcal{C}\left(t_n,t^*,\boldsymbol{\phi}, \mathbf{x}_1, \mathbf{x}_2;\mathbf{s}_n \right) $

$\mathbf{x}_1\leftarrow \mathbf{B}_n$ and $\mathbf{x}_2\leftarrow \boldsymbol{\mathcal{B}}_n$




\For%(\# \textit{if $n=1$ then this for loop will be automatically skipped})
{$i\leftarrow $$n-1$ \KwTo $1$}
{$\mathbf{B}_i \leftarrow H_{\mathbf{s}_{i-1},\mathbf{s}_{i}}   B\left(t_i,t_{i+1},\boldsymbol{\phi}, \mathbf{x}_1;\mathbf{s}_i \right) $
	
$C_i\leftarrow C\left(t_i,t_{i+1},\boldsymbol{\phi}, \mathbf{x}_1;\mathbf{s}_i \right) $

$\boldsymbol{\mathcal{B}}_i \leftarrow H_{\mathbf{s}_{i-1},\mathbf{s}_{i}}   \mathcal{B}\left(t_i,t_{i+1}, \mathbf{x}_2;\mathbf{s}_i \right) $

$\mathcal{C}_i\leftarrow \mathcal{C}\left(t_i,t_{i+1},\boldsymbol{\phi}, \mathbf{x}_1, e^{C_{i+1}}\mathbf{x}_2 ;\mathbf{s}_i \right) $
	
$\mathbf{x}_1\leftarrow \mathbf{B}_i$, and $\mathbf{x}_2\leftarrow \boldsymbol{\mathcal{B}}_i$}
$\mathbf{B}_0 \leftarrow B\left(t_0,t_1,\boldsymbol{\phi}, \mathbf{x}_1;\mathbf{s}_0 \right) $

$C_0\leftarrow C\left(t_0,t_1,\boldsymbol{\phi}, \mathbf{x}_1;\mathbf{s}_0 \right) $ 

$\boldsymbol{\mathcal{B}}_0 \leftarrow \mathcal{B}\left(t_0,t_{1}, \mathbf{x}_2;\mathbf{s}_0 \right) $

$\mathcal{C}_0\leftarrow \mathcal{C}\left(t_0,t_{1},\boldsymbol{\phi}, \mathbf{x}_1, e^{C_{1}}\mathbf{x}_2 ;\mathbf{s}_0 \right) $

$result\leftarrow result+ \psi_{\mathbf{s}_n}\left( \sum_{i=0}^{n} \mathcal{C}_i \exp \{\mathbf{B}_0\cdot \mathbf{y}+ \sum_{j=0}^{i} C_j\}+ \exp\{\mathbf{B}_0\cdot \mathbf{y} +C_0\}\left(\boldsymbol{\mathcal{B}_0 \cdot \mathbf{y}} \right)\right)  $
%\Return{$K\exp\left\lbrace \mathbf{B}_0\cdot \mathbf{y}+\sum_{i=0}^{n} C_i \right\rbrace $}





}


}

\Return{$\left(T-t \right)\frac{result}{100,000}$ }

\label{algo:p2}
\end{algorithm}
\end{document}
